\documentclass{beamer}
\usepackage[utf8]{inputenc}
\usepackage{hyperref}

\usepackage{utopia}

\usetheme{Madrid}
\usecolortheme{default}

%------------------------------------------------------------
\title[JURECA]
{JURECA}
\subtitle{First Modular Supercomputer Worldwide}

\author[Claudio Scheer]
{Claudio~Scheer\inst{1}}

\institute[PUCRS]
{
  \inst{1}%
  Master's Degree in Computer Science\\
  Pontifical Catholic University of Rio Grande do Sul - PUCRS
}

\date[June 2020]
{Parallel Architectures, June 2020}
%------------------------------------------------------------


%------------------------------------------------------------
\AtBeginSection[]
{
  \begin{frame}
    \frametitle{Table of Contents}
    \tableofcontents[currentsection]
  \end{frame}
}
%------------------------------------------------------------


\begin{document}

\frame{\titlepage}

%---------------------------------------------------------
\begin{frame}
  \frametitle{Table of Contents}
  \tableofcontents
\end{frame}
%---------------------------------------------------------


%---------------------------------------------------------
\section{Overview}

\begin{frame}
  \frametitle{Research centre}

  \begin{itemize}
    \item Forschungszentrum Jülich is a interdisciplinary research centre in Germany;
    \item Institute for Advanced Simulation (IAS);
    \item Jülich Supercomputing Centre (JSC);
          \begin{itemize}
            \item Supercomputing centre since 1987;
          \end{itemize}
  \end{itemize}
\end{frame}

\begin{frame}
  \frametitle{Managed supercomputers}
  \begin{itemize}
    \item JUSUF;
    \item JUWELS (position 31\footnote{November 2019 ranking.});
          \begin{itemize}
            \item Helped Google demonstrate the quantum supremacy~\href{https://www.fz-juelich.de/SharedDocs/Pressemitteilungen/UK/EN/2019/2019-10-23-quantum-Supremacy.html}{(source)};
                  \begin{itemize}
                    \item Quantum computer: 200 seconds;
                    \item Fastest supercomputer: 10.000 years;
                  \end{itemize}
          \end{itemize}
    \item JURECA (position 56\footnotemark[\value{footnote}]);
          \begin{itemize}
            \item The name is short for Jülich Research on Exascale Cluster Architectures;
          \end{itemize}
  \end{itemize}
\end{frame}

\begin{frame}
  \frametitle{JURECA}
  \begin{itemize}
    \item 2015-04: begins to operate the cluster;
    \item 2017-11: included a buster module;
    \item First modular supercomputer worldwide~\href{https://www.fz-juelich.de/SharedDocs/Pressemitteilungen/UK/EN/2017/2017-11-13-jureca-booster.html}{(source)};
  \end{itemize}
\end{frame}
%---------------------------------------------------------


%---------------------------------------------------------
\section{Architecture}

\begin{frame}
  \frametitle{JURECA Cluster}
  \includegraphics[width=\textwidth]{./images/jureca-cluster.jpeg}
\end{frame}

\begin{frame}
  \frametitle{JURECA Cluster}
  \begin{itemize}
    \item 1872 compute nodes\footnote{You can see the details \href{https://www.fz-juelich.de/ias/jsc/EN/Expertise/Supercomputers/JURECA/Configuration/Configuration_node.html}{here}.};
          \begin{itemize}
            \item 2 Intel Xeon E5-2680 v3 Haswell CPUs per node;
                  \begin{itemize}
                    \item 2 x 12 cores, 2.5 GHz;
                  \end{itemize}
            \item 75 compute nodes with 2 NVIDIA K80 GPUs;
                  \begin{itemize}
                    \item 4992 CUDA cores;
                    \item 24 GiB GDDR5 memory;
                  \end{itemize}
            \item DDR4 memory (2133 MHz);
                  \begin{itemize}
                    \item 1605 compute nodes with 128 GiB memory;
                    \item 128 compute nodes with 256 GiB memory;
                    \item 64 compute nodes with 512 GiB memory;
                  \end{itemize}
          \end{itemize}
  \end{itemize}
\end{frame}

\begin{frame}
  \frametitle{JURECA Cluster}
  \begin{itemize}
    \item 12 visualization nodes;
          \begin{itemize}
            \item 2 Intel Xeon E5-2680 v3 Haswell CPUs per node;
            \item 2 NVIDIA K40 GPUs per node;
                  \begin{itemize}
                    \item 12 GiB GDDR5 memory;
                  \end{itemize}
            \item 10 nodes with 512 GiB memory;
            \item 2 nodes with 1024 GiB memory;
          \end{itemize}
  \end{itemize}
\end{frame}

\begin{frame}
  \frametitle{Summary - JURECA Cluster}
  \begin{itemize}
    \item 1872 compute nodes;
    \item 12 visualization nodes;
    \item 45.216 CPU cores;
    \item 1.8 (CPU) + 0.44 (GPU) Petaflop per second;
    \item 100 GiB per second storage connection;
  \end{itemize}
\end{frame}

\begin{frame}
  \frametitle{JURECA Buster}
  \includegraphics[width=\textwidth]{./images/jureca-buster.jpeg}
\end{frame}

\begin{frame}
  \frametitle{Summary - JURECA Buster}
  \begin{itemize}
    \item 1640 compute nodes\footnote{You can see the details \href{https://www.fz-juelich.de/ias/jsc/EN/Expertise/Supercomputers/JURECA/Configuration/Configuration_node.html}{here}.};
          \begin{itemize}
            \item 1 Intel Xeon Phi 7250-F Knights Landing CPUs per node;
                  \begin{itemize}
                    \item 68 cores, 1.4 GHz;
                    \item 96 GiB memory plus 16 GiB MCDRAM high-bandwidth memory;
                  \end{itemize}
          \end{itemize}
    \item 111.520 CPU cores;
    \item 5 Petaflop per second;
    \item 100+ GiB per second storage connection;
  \end{itemize}
\end{frame}

\begin{frame}
  \frametitle{Cluster + Buster}
  \begin{itemize}
    \item CentOS 7;
    \item Intel MPI and ParTec MPI;
    \item InfiniBand EDR;
    \item 1,345.28 kW;
  \end{itemize}
\end{frame}
%---------------------------------------------------------


%---------------------------------------------------------
\section{Classifications}

\begin{frame}
  \frametitle{Flynn}
  \begin{itemize}
    \item SISD - Single Instruction, Single Data;
    \item SIMD - Single Instruction, Multiple Data;
    \item MISD - Multiple Instruction, Single Data;
    \item \textbf{MIMD - Multiple Instruction, Multiple Data;}
  \end{itemize}
\end{frame}

\begin{frame}
  \frametitle{Memory sharing}
  \begin{itemize}
    \item Multiprocessor;
    \item \textbf{Multicomputer;}
  \end{itemize}

  % https://youtu.be/LeBRG7dRZ08?t=4860
\end{frame}

\begin{frame}
  \frametitle{Type of memory access}
  \begin{itemize}
    \item UMA - Uniform Memory Access;
    \item NUMA - Non-Uniform Memory Access;
    \item COMA - Cache-Only Memory Architecture;
    \item \textbf{NORMA - Non-Remote Memory Access;}
  \end{itemize}

  % https://github.com/claudioscheer/master-degree/blob/master/parallel-architectures/class-4/annotations.md#academy
\end{frame}

\begin{frame}
  \frametitle{Construction trends}
  \begin{itemize}
    \item PVP - Parallel Vector Processors;
    \item SMP - Symmetric Multiprocessors;
    \item MPP - Massively Parallel Processors;
    \item NOW - Network Of Workstations;
    \item \textbf{COW - Clusters Of Workstations;}
  \end{itemize}

  % https://youtu.be/dsFCeEW0qxU
  % https://github.com/claudioscheer/master-degree/blob/master/parallel-architectures/class-5/annotations.md#annotations
\end{frame}

\begin{frame}
  \includegraphics[width=\textwidth]{./images/classifications.png}
\end{frame}

\begin{frame}
  \frametitle{Dongarra et al. (2003)}

  \begin{columns}
    \column{0.5\textwidth}
    \begin{itemize}
      \item Clustering;
            \begin{itemize}
              \item \textbf{c - commodity cluster;}
              \item m - monolithic system;
            \end{itemize}
    \end{itemize}

    \begin{itemize}
      \item Parallelism;
            \begin{itemize}
              \item t - multithreading;
              \item v - vector;
              \item \textbf{c - communicating sequential processes or message passing;}
              \item s - systolic;
              \item w - VLIW;
              \item h - producer/consumer;
              \item p - parallel processes;
            \end{itemize}
    \end{itemize}

    \column{0.5\textwidth}
    \begin{itemize}
      \item Naming;
            \begin{itemize}
              \item \textbf{d - distributed;}
              \item s - shared;
              \item c - cache coherent;
            \end{itemize}
    \end{itemize}

    \begin{itemize}
      \item Latency;
            \begin{itemize}
              \item c - caches;
              \item v - vectors;
              \item t - multithreaded;
              \item m - processor in memory;
              \item \textbf{p - parcel or message driven split-transaction;}
              \item f - prefetching;
              \item a - explicit allocation;
            \end{itemize}
    \end{itemize}
  \end{columns}
  \bigbreak
  \bigbreak
  Final classification: c/d/c/p.
\end{frame}

\begin{frame}
  \frametitle{Eric E. Johnson (1988)}
  \begin{itemize}
    \item GMSV - Global Memory-Shared Variables;
          \begin{itemize}
            \item Shared memory;
          \end{itemize}
    \item \textbf{DMMP - Distributed Memory-Message Passing;}
          \begin{itemize}
            \item Message passing;
          \end{itemize}
    \item DMSV - Distributed Memory-Shared Variables;
          \begin{itemize}
            \item Hybrid;
          \end{itemize}
    \item GMMP - Global Memory-Message Passing;
  \end{itemize}
\end{frame}
%---------------------------------------------------------


%---------------------------------------------------------
\section{References and resources}

\begin{frame}
  \frametitle{References and resources}
  \begin{itemize}
    \item \href{https://www.fz-juelich.de/ias/jsc/EN/Home/home_node.html}{Jülich Supercomputing Centre};
    \item \href{https://www.youtube.com/watch?v=7h6mYU2HDTA}{Time lapse video of cluster installation};
    \item \href{https://www.youtube.com/watch?v=S84vRFSc-vM}{Time lapse video of booster installation};
  \end{itemize}
\end{frame}
%---------------------------------------------------------

\end{document}